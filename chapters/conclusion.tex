In this thesis we have successfully proposed a generalization of the two concepts \emph{planarly perfect tensors} and \emph{biunitaries}: Perfect tangles in planar algebras.

We  
\begin{itemize}
\item[$\bullet$] showed that they exists in Temperley-Lieb algebras of all orders, as long as the loop parameter satisfies $0<\delta \leq 2$. This was shown by an inductive construction, practically embedding perfect $2$-tangles in any TL algebra of order $n>2$.
\item[$\bullet$] found multiple examples in both the TL planar algebra and in specific trivalent categories, namely cubic ones, one of which was the Haagerup category $H3$.
\item[$\bullet$] wrote code that helps in finding and verifying examples of perfect tangles.
\item[$\bullet$] distilled a conjecture which suggests an infinite family of constructions for larger perfect tangles from smaller ones. In particular these constructions only rely on graphical calculus and keeping track of adjoints, so they would apply to planarly perfect tensors as well.
\end{itemize}
There are a few things and questions that should be investigated in the future.
\begin{itemize}
\item[1.] Do perfect tangles exist in $TL_n$ with $\delta > 2$, for some $n$? 

\item[2.] Since all perfect TL tangles are not in the kernel of the bilinear pairing, they can be seen as nonzero elements in subfactor planar algebras, and still do satisfy the definition of a perfect tangle. It would then be interesting to see how they translate from TL to this more applied setting, and what we may learn from studying them in that light.

\item[3.] The statement of the conjectured constructions boils down to whether these constructed tangles, regarded as honest tangles without labelling, are multilinear maps from the Cartesian product of sets of perfect tangles to a set of perfect tangles. Some effort has gone into and it would hence be nice to see if these maps are actually injective, i.e.\ if distinct tuples of perfect tangles yield distinct larger perfect tangles. In general this is probably false, but at least in the TL case this seems quite plausible, but it turned out to be quite hard to prove.
\end{itemize}

Another project for the future, which is not of immediate concern, is to truly implement planar tangles and by extension planar algebras to aid in calculations. Once this is done, results of, say, the inductive constructions can be presented by a computer. It would then also be much easier to verify the conjectures stated in this thesis for larger examples.

\section*{\textbf{Acknowledgements}}
I would like to thank my advisor, Professor Tobias Osborne, for helpful discussions, consistently clearing up any misconceptions I held, and pushing me to read quite a few papers. I am also very glad I had the chance to learn about the wonderful thing that is planar algebras, and I would like to thank my advisor for talking to Vaughan Jones about my thesis, which brought us \textsf{Theorem \ref{thm:general_existence}} and ultimate led to the conjectures.

My gratitude goes to both Professor Tobias Osborne and Professor Reinhard Werner for refereeing and grading my thesis.

And to anyone who has lent me an ear (or two), and made helpful comments: Thank you!