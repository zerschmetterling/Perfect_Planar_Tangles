%\documentclass[11pt,a4paper]{article}
%\usepackage[utf8]{inputenc}
%\usepackage[english]{babel}
%\usepackage{amsmath, ifthen,caption}
%\usepackage{amsfonts, amssymb,amsthm, thmtools}
%\usepackage{tikz}\usetikzlibrary{arrows, calc,decorations.markings,cd}
%\usepackage[left=3.5cm,right=3.5cm,top=2.5cm,bottom=4cm]{geometry}
%\declaretheorem[style=definition,qed=$\triangle$]{definition}
%%\declaretheorem[style=definition,qed=$\triangle$,numberwithin=section]{definition}
%\declaretheorem[style=remark,qed=$\blacktriangle$,sibling=definition]{remark}
%\declaretheorem[style=plain,sibling=definition]{theorem}
%\declaretheorem[style=plain,sibling=definition]{lemma}
%\declaretheorem[style=plain,sibling=definition]{proposition}
%\declaretheorem[style=plain,sibling=definition]{example}
%
%\newcommand{\mathbf{1}}{\mathrm{id}}
%
%\title{Constructing arbitrarily large perfect tangles from bi\-unitaries}
%\date{}
%\author{Johannes Berger}
%\begin{document}\maketitle
\section{Constructing Large Perfect Tangles}\label{sec:FirstGeneralConstruction}
First a quick informal discussion to sharpen the intuition. Consider the following braid.
\begin{align*}
\begin{tikzpicture}[scale = 0.5]
\braid[number of strands = 5] s_4 s_3 s_4 s_2 s_3 s_4 s_1 s_2 s_3 s_4;
\end{tikzpicture}
\end{align*}
Let us try to explain what is happening here: We see that the string starting at position $i$ on the top is attached to position $5-i$ at the bottom. The first rotation will bring the string attached to the left-most point on the top down, and the one attached to the right-most point on the bottom up. But in this setting this is one and the same string! Multiplying with the adjoint of this rotated version will therefore not hinder our efforts in pulling all strings straight. 

Clearly, all subsequent rotations have similar effects. This realization leads to a somewhat educated guess: 
\begin{center}
Tangles built this way are perfect.
\end{center}
The goal of this section is to give both a well-defined construction and a rigorous proof of the statement.

\bigno
To this end, fix a perfect 2-tangle $T\in P_{2\cdot 2}$, and as before things are unshaded.

\begin{definition}\label{def:T}
Let $\widetilde{T}_2\equiv T$, and define inductively for all $n\geq 3$
\begin{align*}
\widetilde{T}_n \equiv\quad
	\begin{tikzpicture}[baseline=-0.5mm]
		\node[draw] (T1) at (0,0) {$\widetilde{T}_{n-1}$};
		\node[draw] (T2) at (2,0) {$T$};
%		
		\path ($(T1.north) + (-0.2,0)$) edge node[left]{$n-2$} ($(T1.north) + (-0.2,0.5)$)
			($(T1.south)$) edge node[left]{$n-1$} ($(T1.south) + (0,-0.5)$)
			($(T2.north) + (-0.2,0)$) edge ($(T2.north) + (-0.2,0.5)$)
			($(T2.north) + (0.2,0)$) edge ($(T2.north) + (0.2,0.5)$)
			($(T2.south) + (0.2,0)$) edge ($(T2.south) + (0.2,-0.5)$);		
		\draw  ($(T1.north) + (0.2,0)$) .. controls (1.2,1.5) and (0.8,-1.5).. ($(T2.south) + (-0.2,0)$);
	\end{tikzpicture}\quad .
\end{align*}

Also, with $\widehat{T}_2 \equiv T$, we define the flipped version of this:
\begin{align*}
\widehat{T}_n \equiv\quad
	\begin{tikzpicture}[baseline=-0.5mm]
		\node[draw] (T1) at (0,0) {$\widehat{T}_{n-1}$};
		\node[draw] (T2) at (2,0) {$T$};
		\path ($(T1.south) + (-0.2,0)$) edge node[left]{$n-2$} ($(T1.south) + (-0.2,-0.5)$)
			($(T1.north)$) edge node[left]{$n-1$} ($(T1.north) + (0,0.5)$)
			($(T2.south) + (-0.2,0)$) edge ($(T2.south) + (-0.2,-0.5)$)
			($(T2.south) + (0.2,0)$) edge ($(T2.south) + (0.2,-0.5)$)
			($(T2.north) + (0.2,0)$) edge ($(T2.north) + (0.2,0.5)$);		
		\draw  ($(T1.south) + (0.2,0)$) .. controls (1.2,-1.5) and (0.8,1.5).. ($(T2.north) + (-0.2,0)$);
	\end{tikzpicture}\quad,
\end{align*}
which is basically the adjoint of the underlying tangle of $\widetilde{T}_n$, with $T$ inserted into all boxes. Both of these are elements in $P_{2n}$.
\end{definition}

\begin{lemma}\label{lem:T=1}
$\widetilde{T}_n \cdot \widetilde{T}_n^* \propto \mathbf{1}_n$, and a similar result holds for $\widehat{T}_n$.
\begin{proof}
By induction. The base case $n=2$ is clear, since $T$ is perfect. Then suppose that it is true for $n\geq 2$. We get
\begin{align*}
	\begin{tikzpicture}[baseline=-1mm]
		\node[draw] (T1) at (0,0.7) {$\widetilde{T}_{n+1}^*$};
		\node[draw] (T2) at (0,-0.7) {$\widetilde{T}_{n+1}$};
		\path
			(T1.north) edge ($(T1.north) + (0,0.5)$)
			(T2.south) edge ($(T2.south) + (0,-.5)$)
			(T1.south) edge (T2.north);
	\end{tikzpicture}
\,&=
	\begin{tikzpicture}[baseline =-8mm]
		\begin{scope}
			\node[draw] (T1) at (0,0) {$\widehat{T}_{n}^*$};
			\node[draw] (T2) at (2,0) {$T^*$};
			\path ($(T1.south) + (-0.2,0)$) edge ($(T1.south) + (-0.2,-0.5)$)
				($(T1.north)$) edge ($(T1.north) + (0,0.5)$)
				($(T2.south) + (-0.2,0)$) edge ($(T2.south) + (-0.2,-0.5)$)
				($(T2.south) + (0.2,0)$) edge ($(T2.south) + (0.2,-0.5)$)
				($(T2.north) + (0.2,0)$) edge ($(T2.north) + (0.2,0.5)$);		
			\draw  ($(T1.south) + (0.2,0)$) .. controls (1.2,-1.5) and (0.8,1.5).. ($(T2.north) + (-0.2,0)$);
		\end{scope}
		\begin{scope}
			\node[draw] (T1) at (0,-1.7) {$\widetilde{T}_{n}$};
			\node[draw] (T2) at (2,-1.5) {$T$};
			\path ($(T1.north) + (-0.2,0)$) edge ($(T1.north) + (-0.2,0.5)$)
				($(T1.south)$) edge ($(T1.south) + (0,-0.5)$)
				($(T2.north) + (-0.2,0)$) edge ($(T2.north) + (-0.2,0.5)$)
				($(T2.north) + (0.2,0)$) edge ($(T2.north) + (0.2,0.5)$)
				($(T2.south) + (0.2,0)$) edge ($(T2.south) + (0.2,-0.5)$);		
			\draw  ($(T1.north) + (0.2,0)$) .. controls (1.2,0) and (0.8,-3).. ($(T2.south) + (-0.2,0)$);
		\end{scope}
	\end{tikzpicture}
\,\propto
	\begin{tikzpicture}[baseline =-8mm]
		\begin{scope}
			\node[draw] (T1) at (0,0) {$\widehat{T}_{n}^*$};
			\path ($(T1.south) + (0,0)$) edge ($(T1.south) + (0,-0.5)$)
				($(T1.north)$) edge ($(T1.north) + (0,0.5)$)
				($(T1.north) + (1,0.5)$) edge ($(T1.south) + (1,-0.5)$);
		\end{scope}
		\begin{scope}
			\node[draw] (T1) at (0,-1.7) {$\widetilde{T}_{n}$};
			\path ($(T1.north) + (0,0)$) edge ($(T1.north) + (0,0.5)$)
				($(T1.south)$) edge ($(T1.south) + (0,-0.5)$)
				($(T1.north) + (1,0.5)$) edge ($(T1.south) + (1,-0.5)$);
		\end{scope}
	\end{tikzpicture}
\, \stackrel{\mathrm{I.H.}}{\propto} \mathbf{1}_{n+1}
\end{align*}
\end{proof}
\end{lemma}

Also note the following:
\begin{equation}\label{eq:concatenation of T}
\begin{tikzpicture}[baseline=-0.5mm]
	\node[draw] (T1) at (0,0) {$\widetilde{T}_{n}$};
	\node[draw] (T2) at (2,0) {$\widetilde{T}_m$};
%
	\path ($(T1.north) + (-.2,0)$) edge ($(T1.north) + (-0.2,0.5)$)
		($(T1.south)$) edge ($(T1.south) + (0,-0.5)$)
		($(T2.north) + (0,0)$) edge ($(T2.north) + (0,0.5)$)
		($(T2.south) + (0.2,0)$) edge ($(T2.south) + (0.2,-0.5)$);		
	\draw  ($(T1.north) + (0.2,0)$) .. controls (1.2,1.5) and (0.8,-1.5).. ($(T2.south) + (-0.2,0)$);
	\node (eq) at (3.5,0) {$= \widetilde{T}_{n+m-1}$};
\end{tikzpicture}
\end{equation}
(here we are contracting a single strand) and similarly for $\widehat{T}_n$, for all $n,m \geq 2$. 

Consider now the following construction:
\begin{definition}\label{def:B}
Let $A_2 \equiv T = \widetilde{T}_2$, and define for $n\geq 3$
\begin{align*}
A_n \equiv \quad
	\begin{tikzpicture}[baseline = -1mm]
		\node[draw] (T) at (0,0.7) {$\widetilde{T}_{n}$};
		\node[draw] (B) at (0.5, -.7) {$A_{n-1}$};
		\path (T.north) edge ($(T.north) + (0,0.5)$) 
			($(T.south) + (-0.2, 0)$) edge ($(T.south |- B.south) + (-0.2, -0.5)$)
			($(T.south) + (0.2,0)$) edge node[right]{$n-1$} (B.north)
			($(B.south)$) edge ($(B.south |- B.south) + (-0.0, -0.5)$);
	\end{tikzpicture}\quad .
\end{align*}
\end{definition}

Our goal is to show that each $A_n\in P_{2n}$ is perfect. From \textsf{Lemma \ref{lem:T=1}} it is clear that $A_n \cdot A_n^* \propto \mathbf{1}_n$. This is because \text{Lemma \ref{lem:T=1}} asserts that $\widetilde{T}$s occurring together with their adjoints give (something proportional to) identities until only one box is left (and some strings going straight from the top to the bottom), namely $A_2 = T$. But $T$ is perfect, so this gives (something proportional to) the identity.

\bigskip\noindent A basic yet important and beautiful property of the $A_n$ is now stated.
\begin{lemma}\label{lem:Bcommuting}
For all $n\geq 3$
\begin{align*}
	A_n = \quad
	\begin{tikzpicture}[baseline = -1mm]
		\node[draw] (T) at (0,-.7) {$\widehat{T}_{n}$};
		\node[draw] (B) at (0.5, .7) {$A_{n-1}$};
		\path (T.south) edge ($(T.south) + (0,-.5)$) 
			($(T.north) + (-0.2, 0)$) edge ($(T.north |- B.north) + (-0.2, 0.5)$)
			($(T.north) + (0.2,0)$) edge (B.south)
			($(B.north)$) edge ($(B.north |- B.north) + (-0.0, 0.5)$);
	\end{tikzpicture}\quad .
\end{align*}
\begin{proof} By induction. For the base case $n=3$, first remember $A_2=\widetilde{T}_2=\widehat{T}_2 = T$. Then
\begin{align*}
A_3 =\,
	\begin{tikzpicture}[baseline=-2mm]
		\node[draw] (T1) at (0,0.5) {$\widetilde{T}_3$};
		\node[draw] (T2) at (0.2,-0.5) {$T$};
		\foreach \x in {-1,0,1}
			\draw ($(T1.north) + (\x *0.2,0)$) edge ($(T1.north) + (\x *0.2,0.4)$);
		\foreach \x in {-0.1,0.1}{
			\draw ($(T1.south) + (0.1+\x,0)$) -- ($(T2.north) + (\x,0)$);
			\draw ($(T2.south) + (\x,0)$) edge ($(T2.south) + (\x,-0.4)$);
		}
		\draw ($(T1.south) + (-0.2,0)$) edge ($(T1.south |- T2.south) + (-0.2,-0.4)$);
	\end{tikzpicture}
\,&=\,
	\begin{tikzpicture}[baseline=-2mm]
		\node[draw] (T1) at (0,0.5) {$T$};
		\node[draw] (T2) at (0.6,-0.5) {$T$};
		\node[draw] (T3) at (1,0.5) {$T$};
		\foreach \x in {-0.1,0.1}{
			\draw ($(T3.north) + (\x,0)$) edge ($(T3.north) + (\x,0.4)$);
			\draw ($(T2.south) + (\x,0)$) edge ($(T2.south) + (\x,-0.4)$);
		}
		\draw 
			($(T1.north) + (-0.1,0)$) edge ($(T1.north) + (-0.1,0.4)$)
			($(T1.south) + (-0.1,0)$) edge ($(T1.south |- T2.south) + (-0.1,-0.4)$);
		\path 
			($(T1.south) + (+0.1,0)$) edge ($(T2.north) + (-0.1,0)$)
			($(T3.south) + (+0.1,0)$) edge ($(T2.north) + (0.1,0)$);
		\draw ($(T1.north) + (+0.1,0)$) .. controls ($(T1.north) + (0.5,0.5)$) and ($(T3.south) - (0.5,0.5)$) .. ($(T3.south) + (-0.1,0)$);
	\end{tikzpicture}
\,=\,
	\begin{tikzpicture}[baseline=-2mm]
		\node[draw] (T1) at (0,-0.5) {$T$};
		\node[draw] (T2) at (1,-0.5) {$T$};
		\node[draw] (T3) at (.6,0.5) {$T$};
		\foreach \x in {-0.1,0.1}{
			\draw ($(T3.north) + (\x,0)$) edge ($(T3.north) + (\x,0.4)$);
			\draw ($(T2.south) + (\x,0)$) edge ($(T2.south) + (\x,-0.4)$);
		}
		\draw 
			($(T1.south) + (-0.1,0)$) edge ($(T1.south) + (-0.1,-0.4)$)
			($(T1.north) + (-0.1,0)$) edge ($(T1.north |- T3.north) + (-0.1,0.4)$);
		\path 
			($(T1.north) + (+0.1,0)$) edge ($(T3.south) + (-0.1,0)$)
			($(T3.south) + (+0.1,0)$) edge ($(T2.north) + (0.1,0)$);
		\draw ($(T1.south) + (+0.1,0)$) .. controls ($(T1.south) + (0.5,-0.5)$) and ($(T2.north) - (0.5,-0.5)$) .. ($(T2.north) + (-0.1,0)$);
	\end{tikzpicture}
\,=\,
	\begin{tikzpicture}[baseline=-2mm]
		\node[draw] (T1) at (0,-0.5) {$\widehat{T}_3$};
		\node[draw] (T2) at (0.2,0.5) {$T$};
		\foreach \x in {-1,0,1}
			\draw ($(T1.south) + (\x *0.2,0)$) edge ($(T1.south) + (\x *0.2,-0.4)$);
		\foreach \x in {-0.1,0.1}{
			\draw ($(T1.north) + (0.1+\x,0)$) -- ($(T2.south) + (\x,0)$);
			\draw ($(T2.north) + (\x,0)$) edge ($(T2.north) + (\x,0.4)$);
		}
		\draw ($(T1.north) + (-0.2,0)$) edge ($(T1.north |- T2.north) + (-0.2,0.4)$);
	\end{tikzpicture}\quad,
\end{align*}
as required. Now suppose that the claim is true up to $n-1$. Then we have
\begin{align*}
A_n &=\,
	\begin{tikzpicture}[baseline=-2mm]
		\node[draw] (T1) at (0,0.8) {$\widetilde{T}_n$};
		\node[draw] (T2) at (0.6,-0.8) {$A_{n-1}$};
		\draw ($(T1.north) + (0 *0.2,0)$) edge ($(T1.north) + (0 *0.2,0.4)$);
		\draw ($(T1.south) + (0.1+0,0)$) -- ($(T2.north) + (0,0)$);
		\draw ($(T2.south) + (0,0)$) edge ($(T2.south) + (0,-0.4)$);
		\draw ($(T1.south) + (-0.2,0)$) edge ($(T1.south |- T2.south) + (-0.2,-0.4)$);
	\end{tikzpicture}
\,=\,
	\begin{tikzpicture}[baseline=-2mm]
		\node[draw] (T1) at (0,0) {$T$};
		\node[draw] (Tt) at (1,1) {$\widetilde{T}_{n-1}$};
		\node[draw] (B) at (1,-1) {$A_{n-1}$};
%
		\draw ($(T1.south) + (-0.1,0)$)	edge ($(T1.south |- B.south) + (-0.1,-0.4)$);
		\draw ($(T1.north) + (-0.1,0)$)	edge ($(T1.north |- Tt.north) + (-0.1,0.4)$);
		\path
			($(T1.north) + (0.1,0)$) edge ($(Tt.south) + (-0.3,0)$)
			($(T1.south) + (0.1,0)$) edge ($(B.north) + (-0.3,0)$)
			($(B.north) + (+0.3,0)$)edge ($(Tt.south) + (0.3,0)$);
		\path
			(B.south) edge ($(B.south) + (-0,-0.4)$)
			(Tt.north) edge ($(Tt.north) + (-0,0.4)$);
	\end{tikzpicture}
\,\stackrel{\mathrm{I.H.}}{=}\,
	\begin{tikzpicture}[baseline=-2mm]
		\node[draw] (T1) at (0,0) {$T$};
		\node[draw] (Tt) at (1,1) {$\widetilde{T}_{n-1}$};
		\node[draw] (That) at (1,-1) {$\widehat{T}_{n-1}$};
		\node[draw] (B) at (1.3,0) {$A_{n-2}$};
%
		\draw ($(T1.south) + (-0.1,0)$)	edge ($(T1.south |- That.south) + (-0.1,-0.4)$);
		\draw ($(T1.north) + (-0.1,0)$)	edge ($(T1.north |- Tt.north) + (-0.1,0.4)$);
		\path
			($(T1.north) + (0.1,0)$) edge ($(Tt.south) + (-0.3,0)$)
			($(T1.south) + (0.1,0)$) edge ($(That.north) + (-0.3,0)$)
			($(That.north) + (+0.3,0)$)edge ($(B.south) + (0.0,0)$)
			($(B.north) + (+0.0,0)$)edge ($(Tt.south) + (0.3,0)$);
		\path
			(That.south) edge ($(That.south) + (-0,-0.4)$)
			(Tt.north) edge ($(Tt.north) + (-0,0.4)$);
	\end{tikzpicture}\\[1em]
&=\,
	\begin{tikzpicture}[baseline=-2mm]
		\node[draw] (T1) at (0,0) {$T$};
		\node[draw] (Tt) at (1,1) {$A_{n-1}$};
		\node[draw] (B) at (1,-1) {$\widehat{T}_{n-1}$};
%
		\draw ($(T1.south) + (-0.1,0)$)	edge ($(T1.south |- B.south) + (-0.1,-0.4)$);
		\draw ($(T1.north) + (-0.1,0)$)	edge ($(T1.north |- Tt.north) + (-0.1,0.4)$);
		\path
			($(T1.north) + (0.1,0)$) edge ($(Tt.south) + (-0.3,0)$)
			($(T1.south) + (0.1,0)$) edge ($(B.north) + (-0.3,0)$)
			($(B.north) + (+0.3,0)$)edge ($(Tt.south) + (0.3,0)$);
		\path
			(B.south) edge ($(B.south) + (-0,-0.4)$)
			(Tt.north) edge ($(Tt.north) + (-0,0.4)$);
	\end{tikzpicture}
\,=\,
	\begin{tikzpicture}[baseline = -2mm]
		\node[draw] (T) at (0,-.7) {$\widehat{T}_{n}$};
		\node[draw] (B) at (0.5, .7) {$A_{n-1}$};
		\path (T.south) edge ($(T.south) + (0,-.5)$) 
			($(T.north) + (-0.2, 0)$) edge ($(T.north |- B.north) + (-0.2, 0.5)$)
			($(T.north) + (0.2,0)$) edge (B.south)
			($(B.north)$) edge ($(B.north |- B.north) + (-0.0, 0.5)$);
	\end{tikzpicture}
\end{align*}
\end{proof}
\end{lemma}

This lemma allows us to switch between two ways of writing $A_n$, which ever way is more comfortable to work with in a given situation. The statement $A_n^*\cdot A_n\propto\mathbf{1}_n$, for example, is now a simple corollary of \textsf{Lemma \ref{lem:Bcommuting}} and the remark made after \textsf{Definition \ref{def:B}}.

One thing we should note is that if we interpret $\widehat{\phantom{T}}_n$ and $\widetilde{\phantom{T}}_n$ as $n$-tangles with $n-1$ internal boxes, then
\begin{align*}
\widetilde{T}_n^* = \widehat{T^*}_n,
\end{align*}
so that $\widetilde{T}_n\cdot \widehat{T^*}_n\propto\mathbf{1}_n$ by \textsf{Lemma \ref{lem:T=1}}.

To prove that the rotations of $A_n$ are also `unitary', we still need to record a few more facts, like the following lemma.
\begin{lemma}\label{lem:half rotated  TT}
Let $1\leq l < n-1$, $n\geq 3$. Then
\begin{align*}
	\begin{tikzpicture}[baseline=-1mm]
			\node[draw] (Tdag) at (0,.7) {$\widetilde{T}_n^*$};
			\node[draw] (T) at (0,-.7) {$\widetilde{T}_n$};
			\path (Tdag.north) edge ($(Tdag.north) + (0,0.5)$)
				(T.south) edge ($(T.south) + (0,-.5)$)
				($(T.north) + (0.2,0)$) edge node[right]{$n-l$} ($(Tdag.south) + (0.2,0)$)
				($(T.north) + (-0.6,0)$) edge node[left] {$l$} ($(T.south) + (-0.6,-0.5)$)
				($(Tdag.south) + (-0.6,0)$) edge node[left] {$l$} ($(Tdag.north) + (-0.6,0.5)$);
			\draw ($(T.north) + (-.2,0)$) arc [start angle=0, end angle=180, radius=0.2cm];
			\draw ($(Tdag.south) + (-.2,0)$) arc [start angle=0, end angle=-180, radius=0.2cm];
	\end{tikzpicture}
\, \propto \,
	\begin{tikzpicture}[baseline=-1mm]
			\node[draw] (Tdag) at (0,.7) {$\widetilde{T}_{l+1}^*$};
			\node[draw] (T) at (0,-.7) {$\widetilde{T}_{l+1}$};
			\path (Tdag.north) edge ($(Tdag.north) + (0,0.5)$)
				(T.south) edge ($(T.south) + (0,-.5)$)
				($(T.north) + (0.2,0)$) edge node[right]{$1$} ($(Tdag.south) + (0.2,0)$)
				($(T.north) + (-0.6,0)$) edge node[left] {$l$} ($(T.south) + (-0.6,-0.5)$)
				($(Tdag.south) + (-0.6,0)$) edge node[left] {$l$} ($(Tdag.north) + (-0.6,0.5)$);
			\draw ($(T.north) + (-.2,0)$) arc [start angle=0, end angle=180, radius=0.2cm];
			\draw ($(Tdag.south) + (-.2,0)$) arc [start angle=0, end angle=-180, radius=0.2cm];
			\draw ($(T.south) + (,-0.5)$) edge node[right]{$n-(l+1)$} ($(Tdag.north) + (1,0.5)$);
	\end{tikzpicture}\,,
\end{align*}
where a string with a 0 next to it is interpreted as no string at all.
\begin{proof}
A computation with careful consideration of the number of strings, and a trick involving \eqref{eq:concatenation of T}, shows the result. We will not write the string count next to each string, so you should definitely do that to help you understand what's going on.

Fix any $n\geq 3$. Here is the calculation:
\begin{align*}
	\begin{tikzpicture}[baseline=-1mm]
			\node[draw] (Tdag) at (0,.7) {$\widetilde{T}_n^*$};
			\node[draw] (T) at (0,-.7) {$\widetilde{T}_n$};
			\path (Tdag.north) edge ($(Tdag.north) + (0,0.5)$)
				(T.south) edge ($(T.south) + (0,-.5)$)
				($(T.north) + (0.2,0)$) edge node[right]{$n-l$} ($(Tdag.south) + (0.2,0)$)
				($(T.north) + (-0.6,0)$) edge node[left] {$l$} ($(T.south) + (-0.6,-0.5)$)
				($(Tdag.south) + (-0.6,0)$) edge node[left] {$l$} ($(Tdag.north) + (-0.6,0.5)$);
			\draw ($(T.north) + (-.2,0)$) arc [start angle=0, end angle=180, radius=0.2cm];
			\draw ($(Tdag.south) + (-.2,0)$) arc [start angle=0, end angle=-180, radius=0.2cm];
	\end{tikzpicture}
\, &= \,
	\begin{tikzpicture}[baseline=-1mm]
		\coordinate (yValue) at (0,1);
		\coordinate (xValue) at (0.7,0);
		\node[draw] (LD) at ($(yValue)-(xValue)$) {$\widetilde{T}_{l+1}^*$};
		\node[draw] (ND) at ($(yValue)+(xValue)$) {$\widetilde{T}_{n-l}^*$};
		\node[draw] (L) at ($(0,0)-(xValue)-(yValue)$) {$\widetilde{T}_{l+1}$};
		\node[draw] (N) at ($(xValue)-(yValue)$) {$\widetilde{T}_{n-l}$};
%
		\coordinate (yMin) at ($(L.south) + (0,-0.4)$);
		\coordinate (yMax) at ($(LD.north) + (0,0.4)$);
		\foreach \x in {-0.3 -0.4, 0.3}
			\draw ($(L.north) + (\x,0)$) arc[start angle=180, end angle=0, radius=2mm];
		\draw ($(LD.south) + (0.3+0.4,0)$) -- ($(ND.north) + (-0.3-0.4,0)$) arc[start angle=180, end angle=0, radius=2mm];
		\foreach \x in {-0.3 -0.4, 0.3}
			\draw ($(LD.south) + (\x,0)$) arc[start angle=-180, end angle=0, radius=2mm];
		\draw ($(L.north) + (0.3+0.4,0)$) -- ($(N.south) + (-0.3-0.4,0)$) arc[start angle=-180, end angle=0, radius=2mm];
		\foreach \x in {LD.north, ND.north}
			\draw (\x) -- (\x |- yMax);
		\draw ($(LD.south)+(-0.3-0.4,0)$) -- ($(LD.south |- yMax)+(-0.3-0.4,0)$);
		\foreach \x in {L.south, N.south}
			\draw (\x) -- (\x |- yMin);
		\draw ($(L.north)+(-0.3-0.4,0)$) -- ($(L.north |- yMin)+(-0.3-0.4,0)$);
		\draw (ND.south) -- (N.north);
	\end{tikzpicture}
\, \propto \,
	\begin{tikzpicture}[baseline=-1mm]
		\coordinate (yValue) at (0,1);
		\coordinate (xValue) at (0.7,0);
		\node[draw] (LD) at ($(yValue)-(xValue)$) {$\widetilde{T}_{l+1}^*$};
		\node[draw] (L) at ($(0,0)-(xValue)-(yValue)$) {$\widetilde{T}_{l+1}$};
%
		\coordinate (yMin) at ($(L.south) + (0,-0.4)$);
		\coordinate (yMax) at ($(LD.north) + (0,0.4)$);
		\foreach \x in {-0.3 -0.4}
			\draw ($(L.north) + (\x,0)$) arc[start angle=180, end angle=0, radius=2mm];
		\foreach \x in {-0.3 -0.4}
			\draw ($(LD.south) + (\x,0)$) arc[start angle=-180, end angle=0, radius=2mm];
		\foreach \x in {LD.north}
			\draw (\x) -- (\x |- yMax);
		\draw ($(LD.south)+(-0.3-0.4,0)$) -- ($(LD.south |- yMax)+(-0.3-0.4,0)$);
		\foreach \x in {L.south}
			\draw (\x) -- (\x |- yMin);
		\draw ($(L.north)+(-0.3-0.4,0)$) -- ($(L.north |- yMin)+(-0.3-0.4,0)$);
		\draw ($(LD.south)+(0.3,0)$) -- ($(L.north)+(0.3,0)$);
		\draw ($(yMin) + (1.2,0)$) -- node[right]{$n-l-1$} ($(yMax) + (1.2,0)$);
	\end{tikzpicture}
\end{align*}
The first equality is the trick, the second (well, not equality but proportionality) uses \textsf{Lemma \ref{lem:T=1}}.
\end{proof}
\end{lemma}

If you read the previous lemma carefully you will probably have noticed that we excluded the $l=n-1$ case. One reason for this is that we can postpone that part until we are in our main theorem, where we solve it by invoking \textsf{Lemma \ref{lem:Bcommuting}}. The other reason is that the formula would be ill-defined, since $n-(n-1) = 1$, but $\widetilde{T}_1$ is undefined.


The following proposition is straightforward, but the result is very useful in one of the last steps in the proof of the main theorem.
\begin{proposition}\label{prop:T rotated n-1 times}
For all $n\geq 2$
\begin{align*}
	\rot^{n-1}\widetilde{T}_n\cdot \left(\rot^{n-1}\widetilde{T}_n\right)^* = 
	\begin{tikzpicture}[baseline=-1mm]
			\node[draw] (Tdag) at (0,.7) {$\widetilde{T}_n^*$};
			\node[draw] (T) at (0,-.7) {$\widetilde{T}_n$};
			\path
				($(Tdag.north) + (-0.2,0)$) edge ($(Tdag.north) + (-0.2,0.5)$)
				($(T.south) + (-0.2,0)$) edge ($(T.south) + (-0.2,-0.5)$)
				($(T.north) + (0.2,0)$) edge ($(Tdag.south) + (0.2,0)$)
				($(T.north) + (-0.6,0)$) edge node[left] {$n-1$} ($(T.south) + (-0.6,-0.5)$)
				($(Tdag.south) + (-0.6,0)$) edge node[left] {$n-1$} ($(Tdag.north) + (-0.6,0.5)$);
			\draw ($(T.north) + (-.2,0)$) arc [start angle=0, end angle=180, radius=0.2cm];
			\draw ($(T.south) + (.2,0)$) arc [start angle=-180, end angle=0, radius=0.2cm] -- node[right]{$n-1$}
				 ($(Tdag.north) + (.6,0)$) arc [start angle=0, end angle=180, radius=0.2cm];
			\draw ($(Tdag.south) + (-.2,0)$) arc [start angle=0, end angle=-180, radius=0.2cm];
	\end{tikzpicture}
\, \propto \mathbf{1}_n
\end{align*}
\end{proposition}
\begin{proof}Omitted. An exercise in mathematical induction for the reader.
\end{proof}

We are now ready to prove our main result.
\begin{theorem}\label{thm:general_existence}
For all $A_n$ the following are true.
\begin{itemize}
	\item[\emph{\text{(i)}}] For all $1\leq l < n$ we have
		\begin{align*}
		\mathrm{rot}^l A_n \cdot \mathrm{rot}^{-l} A_n^* \propto \mathbf{1}_n.
		\end{align*}
	\item[\emph{\text{(ii)}}] $A_n$ is perfect.
\end{itemize}
\begin{proof}
We will first show (i), then (i)$\implies$(ii). The second part is straigthforward.

For (i), first note that this is trivially true for $A_2$, and that proving it for $A_3$ is a basic computation. The base case is thus covered. Now suppose it is true for $n-1$, and consider first $l< n-1$. Then
\begin{align*}
\mathrm{rot}^l A_n \cdot \mathrm{rot}^{-l} A_n^*
&=\,
	\begin{tikzpicture}[baseline=-1mm]
		\node[draw] (BD) at (1,2) {$A_{n-1}^*$};
		\node[draw] (TD) at (0,0.8) {$\widetilde{T}_n^*$};
		\node[draw] (T) at (0,-0.8) {$\widetilde{T}_n$};
		\node[draw] (B) at (1, -2) {$A_{n-1}$};
		\coordinate (yMax) at ($(BD.north)+(0,0.4)$);
		\coordinate (yMin) at ($(B.south)+(0,-0.4)$);
		\draw 
			($(B.south)+(0.4,0)$) arc[start angle=-180, end angle=0, radius=2mm] -- node[right] {$l$}
			($(BD.north)+(0.8,0)$) arc[start angle=0, end angle=180, radius=2mm];
		\path 
			($(TD.north)+(0.2,0)$) edge ($(BD.south)+(-0.4,0)$)
			($(TD.south)+(0.2,0)$) edgenode[right] {$n-l$}  ($(T.north)+(0.2,0)$)
			($(T.south)+(0.2,0)$) edge ($(B.north)+(-0.4,0)$);
		\draw
			($(TD.south)+(-0.2,0)$) arc[start angle=0, end angle=-180, radius=2mm] edge node[left] {$l$} ($(TD.south |- yMax)+(-0.6,0)$);
		\draw
			($(T.north)+(-0.2,0)$) arc[start angle=0, end angle=180, radius=2mm] edge node[left] {$l$} ($(TD.south |- yMin)+(-0.6,0)$);
		\path
			($(BD.north)+(-0.2,0)$) edge ($(BD.north |- yMax)+(-0.2,0)$)
			($(TD.north)+(-0.2,0)$) edge ($(TD.north |- yMax)+(-0.2,0)$)
			($(T.south)+(-0.2,0)$) edge ($(T.south |- yMin)+(-0.2,0)$)
			($(B.south)+(-0.2,0)$) edge ($(B.south |- yMin)+(-0.2,0)$);
	\end{tikzpicture}
\,=\,
	\begin{tikzpicture}[baseline=-1mm]
		\node[draw] (BD) at (1,2) {$A_{n-1}^*$};
		\node[draw] (TD) at (0,0.8) {$\widetilde{T}_{l+1}^*$};
		\node[draw] (T) at (0,-0.8) {$\widetilde{T}_{l+1}$};
		\node[draw] (B) at (1, -2) {$A_{n-1}$};
		\coordinate (yMax) at ($(BD.north)+(0,0.4)$);
		\coordinate (yMin) at ($(B.south)+(0,-0.4)$);
		\draw 
			($(B.south)+(0.4,0)$) arc[start angle=-180, end angle=0, radius=2mm] -- node[right] {$l$}
			($(BD.north)+(0.8,0)$) arc[start angle=0, end angle=180, radius=2mm];
		\path 
			($(TD.north)+(0.2,0)$) edge node[right]{$l$} ($(BD.south)+(-0.4,0)$)
			($(TD.south)+(0.2,0)$) edge node[right] {$1$}  ($(T.north)+(0.2,0)$)
			($(T.south)+(0.2,0)$) edge ($(B.north)+(-0.4,0)$);
		\draw
			($(BD.south)+(+0.4,0)$) edge node[above, rotate=90]{{\small$n-1-l$}}  ($(B.north)+(0.4,0)$);
		\draw
			($(TD.south)+(-0.2,0)$) arc[start angle=0, end angle=-180, radius=2mm] edge node[left] {$l$} ($(TD.south |- yMax)+(-0.6,0)$);
		\draw
			($(T.north)+(-0.2,0)$) arc[start angle=0, end angle=180, radius=2mm] edge node[left] {$l$} ($(TD.south |- yMin)+(-0.6,0)$);
		\path
			($(BD.north)+(-0.2,0)$) edge ($(BD.north |- yMax)+(-0.2,0)$)
			($(TD.north)+(-0.2,0)$) edge ($(TD.north |- yMax)+(-0.2,0)$)
			($(T.south)+(-0.2,0)$) edge ($(T.south |- yMin)+(-0.2,0)$)
			($(B.south)+(-0.2,0)$) edge ($(B.south |- yMin)+(-0.2,0)$);
	\end{tikzpicture}\\[2em]
\, &\propto \,
	\begin{tikzpicture}[baseline=-1mm]
			\node[draw] (Tdag) at (0,.7) {$\widetilde{T}_{l+1}^*$};
			\node[draw] (T) at (0,-.7) {$\widetilde{T}_{l+1}$};
			\path
				($(Tdag.north) + (-0.2,0)$) edge ($(Tdag.north) + (-0.2,0.5)$)
				($(T.south) + (-0.2,0)$) edge ($(T.south) + (-0.2,-0.5)$)
				($(T.north) + (0.2,0)$) edge ($(Tdag.south) + (0.2,0)$)
				($(T.north) + (-0.6,0)$) edge node[left] {$l$} ($(T.south) + (-0.6,-0.5)$)
				($(Tdag.south) + (-0.6,0)$) edge node[left] {$l$} ($(Tdag.north) + (-0.6,0.5)$);
			\draw ($(Tdag.north) + (1.5,0.5)$) -- node[right]{$n-1-l$} ($(T.south) + (1.5,-0.5)$);
%			
			\draw ($(T.north) + (-.2,0)$) arc [start angle=0, end angle=180, radius=0.2cm];
			\draw ($(T.south) + (.2,0)$) arc [start angle=-180, end angle=0, radius=0.2cm] -- node[right]{$l$}
				 ($(Tdag.north) + (.6,0)$) arc [start angle=0, end angle=180, radius=0.2cm];
			\draw ($(Tdag.south) + (-.2,0)$) arc [start angle=0, end angle=-180, radius=0.2cm];
	\end{tikzpicture}\\[2em]
&\propto \mathbf{1}_n
\end{align*}
What's happening here is the following. The first step is writing out the definition. The second is applying \textsf{Lemma \ref{lem:half rotated  TT}}, after which we are at the induction step: it's true that $\mathrm{rot}^l A_{n-1} \cdot \mathrm{rot}^{-l} A_{n-1}^* \propto \mathbf{1}_{n-1}$. Note that the largest value of $l$ here is $(n-1)-1$.
Last but not least, we invoke \textsf{Proposition \ref{prop:T rotated n-1 times}}, which gives the desired result.

For $l=n-1$ note that the rotation bends \emph{all} legs of $A_{n-1}$ up- and all legs of $A_{n-1}^*$ downwards, thereby contracting the $A$s along half of their legs. That is we have the product $A_{n-1}^*\cdot  A_{n-1}$, which, as we have argued before, is proportional to $\mathbf{1}_{n-1}$. Then all that's left is applying \textsf{Proposition \ref{prop:T rotated n-1 times}} one time and the proof of part (i) is finished.

\bigskip
Part (ii) of the theorem follows from \textsf{Proposition \ref{Prop:ROT INVROT}}, i.e.\
\begin{align*}
0\leq l <k : \mathrm{rot}^{k+l} T\cdot\mathrm{rot}^{-(k+l)}T^* \propto \mathbf{1}_k \quad\iff\quad \mathrm{rot}^{-l}T^* \cdot \mathrm{rot}^{l} T\propto \mathbf{1}_k,
\end{align*}
and an argument about the validity of `horizontally flipped' versions of \textsf{Lemma \ref{lem:half rotated  TT}} and \textsf{Proposition \ref{prop:T rotated n-1 times}}, after seeing that the base case is (almost trivially) true.\footnote{In the TL planar algebra, one could also make a handwavy argument that actually follows from $\rot^n A_n = A_n$, like so. Observe that $T\in TL_2$ is necessarily symmetric under rotation by 2 clicks. Then we have $\rot^n\widetilde{T}_n = \widetilde{T}_n$, and consequently the same holds for $A_n$ because it's made up of $\widetilde{T}$s.} 
%about the symmetry of $A_n$ shown in Lemma \ref{lem:Bcommuting}, as well as interpreting $\widetilde{\phantom{T}}_n$ as a tangle with $n-1$ internal boxes.
\end{proof}
\end{theorem}

These perfect tangles constructed in this section are certainly very pretty. In \figref{fig:examples_general_construction} we see a few examples which are in $P_{2n}$ for $n=3,4,5,6$, respectively. The perfect tangle $T\in TL_{2\cdot 2}$ is represented as a vertex and should be interpreted as being in standard form [\emph{cf.\ the very second page of this thesis}].
\begin{figure}[!htp]\centering
\captionsetup{width=.7\linewidth}
\begin{tikzpicture}[inner sep=0.5mm,scale=1]
	\foreach \order in {1,...,4}{
		\begin{scope}[xshift=1/2*\order*\order cm]
			\coordinate (yMax) at (0, .7+ 0.5*\order);
			\coordinate (yMin) at (0,-.7-.5*\order);
			\foreach \x in {0,...,\order}{
				\foreach \y in {0,...,\x}{		
					\node[draw, fill,circle] (x\x) at (.8*\x,.5*\x -1*\y) {};
					\ifthenelse{\NOT \x = \order}{
						\draw (x\x) -- ($(x\x) + (.8, .5 - 1)$);
						\draw (x\x) -- ($(x\x) + (.8, -.5 + 1)$);
						\ifthenelse{\y=\x}{
							\ifthenelse{\y=0}{\draw (x\x) -- (x\x |- yMax);}{}
							\draw (x\x) -- (x\x |- yMin);}{\ifthenelse{\y=0}{\draw (x\x) -- (x\x |- yMax);}{}
						}
					}{
						\ifthenelse{\y=0}{
							\draw (x\x) -- ($(x\x) + (0,-\order)$);
							\draw ($(x\x.north) + (0.05,-0.05)$) -- ($(x\x.north |- yMax) + (0.05, 0)$);
							\draw ($(x\x.north) - (0.05, 0.05)$) -- ($(x\x.north |- yMax) - (0.05,0)$);
						}{
							\ifthenelse{\y=\x}{
							\draw ($(x\x.south) - (0.05,-0.05)$) -- ($(x\x.south |- yMin) - (0.05, 0)$);
							\draw ($(x\x.south) +(0.05, 0.05)$) -- ($(x\x.south |- yMin) + (0.05,0)$);
							}{}
						}
					}
				}
			}
			\ifthenelse{\NOT \order=4}{\node (comma) at (0.2+.8*\order,0) {$,$};}{}
		\end{scope}
	}
\end{tikzpicture}
\caption[Examples of perfect tangles obtained by the iterative construction proved in \textsf{Theorem \ref{thm:general_existence}}]{The perfect tangles $A_3,A_4,A_5,A_6$. Vertices represent the generating perfect 2-tangle $T$ in standard form.}
\label{fig:examples_general_construction}
\end{figure}


It is clear that what this theorem actually does is that it gives us for each $n$ an $n$-tangle with $\frac{n(n-1)}{2}$ internal 2-boxes, such that if we fix a perfect 2-tangle and insert a copy into each box the resulting element is perfect.

%If we let $P_n$ denote the set of perfect $n$-tangles with
%\begin{align*}
%T\in P_n \quad \implies \quad \rot^l(T) \not \in P_n\qquad \forall 0 < l < 2n,
%\end{align*}

In the next section we will argue (but not show) that this construction, among others, works for any planar algebra with perfect $2$-tangles. Letting $\mathcal{P}_n$ denote the set of perfect $n$-tangles --- including adjoints and rotations of tangles already in $\mathcal{P}_n$ --- then \textsf{Theorem \ref{thm:general_existence}} shows a lower bound on the size of this set:
\begin{align}\label{eq:lower_bound no1}
\lvert \mathcal{P}_2 \rvert \leq \lvert \mathcal{P}_n \rvert \quad \forall n\geq 2.
\end{align}
So whenever $\mathcal{P}_2\neq \emptyset$, the general existence of perfect tangles in spaces indexed by an even color is established.

%\section{Going even further: Raising the lower bound}
%The fact that the procedure of constructing $A_n$ yields an $n$-tangle, together with the observation of how the proof of Theorem \ref{thm:general_existence} actually works, raises some hope of a further generalization --- an even better result. 
%
%The section is devoted to stating \emph{and} proving this more general version, effectively raising the lower bound given in equation (\ref{eq:lower_bound no1}). We will first state it as a (justified) conjecture, which will then, over the course of this section, be proven.
%
%\bigno
%Take the $n$-tangle $A_n$ from the previous section, and `remove' all $T$s.  By the very definition of planar tangles, we are then left with a multilinear map, which we shall denote by
%\begin{align*}
%\widetilde{B}_n:\bigtimes_{i=1}^{\frac{n(n-1)}{2}} TL_2 \rightarrow TL_n.
%\end{align*}
%Because of the high symmetry of $A_n$, and looking back at the proof of the big theorem in the last section, we are led to the following:
%
%
%\begin{center}\boxed{
%	\begin{minipage}{0.85\textwidth}\textbf{\textsf{Educated Hunch.}}\\
%		Using the notation $P_n$ for the set of all perfect tangles, we first claim
%	\begin{align*}\label{conjecture:1}\tag{\textsf{Hunch 1}}
%		\widetilde{B}_n\left( P_2^{\frac{n(n-1)}{2}} \right) \subset P_n,\footnotemark
%	\end{align*}
%	that is any $\frac{n(n-1)}{2}$-tuple of perfect 2-tangles gets mapped to a perfect $n$-tangle.
%
%	\bigno
%	The second claim that we make is: $\widetilde{B}_n$ restricted to perfect 2-tangles is bijective onto its image. That is we allege that the map
%	\begin{align*}\label{conjecture:2} \tag{\textsf{Hunch 2}}
%		\at{\widetilde{B}_n}{P_2^{\frac{n(n-1)}{2}}} : P_2^{\frac{n(n-1)}{2}} \longrightarrow P_n%TL_n
%	\end{align*}
%	is injective, thus effectively raising the lower bound (\ref{eq:lower_bound no1}) on the number of \emph{distinct} perfect $n$-tangles from $\lvert P_2 \rvert$ to
%	\begin{align*}
%		\lvert P_2 \rvert^{\frac{n(n-1)}{2}} \leq \lvert P_n \rvert
%	\end{align*}
%	\end{minipage}
%}\end{center}\footnotetext{If $S$ is a set and $k$ is a natural number, then $S^k$ is the $k$th Cartesian power of $S$.}
%
%\ref{conjecture:1}
%It is important to emphasize that $TL_n\subset \mathcal{C}_{2n}$ (the hom-space $\mathcal{C}[1,X^{\otimes 2n}]$). By construction, any $A_n$ can thus be interpreted as a morphism in a trivalent category, which, in particular, is perfect. So this section not only gave us a ton of perfect tangles, but also more than enough perfect morphism!
%\end{document}
